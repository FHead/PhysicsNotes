\documentclass{fheadnote}

\usepackage{amsmath}
\usepackage{amsfonts}
\usepackage{amssymb}
\usepackage{makeidx}
\usepackage{graphicx}
\usepackage{url}
\usepackage{verbatim}
\usepackage{textcomp}
\usepackage{footnote}
\usepackage{hyperref}

\begin{document}

\begin{titlepage}

   \fheadreport{2012/001}
   \date{\today}

   \title{SUSY 2012 report}

   \begin{Authlist}
      Yi Chen
      \Instfoot{Institute}{California Institute of Technology}
   \end{Authlist}

   % \collaboration{FHead collaboration}

   \begin{abstract}
      Report of the SUSY 2012 conference.
   \end{abstract}

\end{titlepage}

\setcounter{page}{2}

\section{Introduction}

I participated in SUSY 2012 conference in Beijing, Aug. 13-19 2012.
This short report is a summary of the interesting points I've noticed.

\section{Interesting points}

\subsection{DM experiments}

XENON 100 collaboration has a new result in 2012.
Experimentally with Xenon target, they'll have two signals, one from ionized + drifted electron,
one from prompt photon.
3D spatial reconstruction is also possible, and is important in background rejection.
XENON 1T hope to start soon.

LUX experiment is under construction.  It's also Xenon target.  100 kg fiducial volume.
1.5 km underground + immersed in water for better background rejection.

KIMS experiment is a small experiment in Korea aim to measure annual modulation.
Target is CsI crystal (solid), and they have collected 75.53 ton-days of data.
For the result, fit with exponential (internal radiation source) + constant (external) + modulation.
No significant modulation found, in contradiction with previous experiment.

In China, they are launching several experiments.
TANSUO experiment is space-based (like AMS), expected to have very good performance.
CJPL underground detector.  PandaX, Xenon detector similar to XENON and LUX.
CDEX: Ge solid detector.

AMS presentation shows a lot of good performance plots, but no physics results yet.

\subsection{Models for DM}

One topic that comes out is the dynamical dark matter.
Instead of stability of some new paritcle, imagine an ensemble of new particles.
For example the di-jet mass distribution will have weird shapes.
It's still very far from mature, so we'll have to wait before we can start searching for this.

Maybe Higgs is a psuedo Nambu-Goldstone boson due to breaking of larger symmetry.
In this model (+some other assumptions), we can put in a dark matter candidate
which has contact interactions to higgs and fermions.

Maybe DM violated iso-spin (couples to up and down quark differently).
We can measure this via interference in $q\bar{q} \rightarrow q\bar{q}$ diagrams,
where DM pair production can come from initial quark or final quarks.
The experimental signature to look for is mono-photon and mono-jet and compare.
Needs huge amount of data, even without systematic errors.
Not practical at the moment.

Axion-like particles (hidden-photons, etc.).
The universe is a bit too transparent to high energy photons.
See also D. Horns, M. Meyer (2012) for this opacity anomaly.
Maybe photons got converted to axion-like particles, travel a bit, and convert back.
Toy model studied, where space is split into different regimes of uniform magnetic field
of random directions.  If this is true we should see the spectrum getting very noisy.
As a result if we look at the variance around the smoothed spectrum line,
there should be a sign.

Millicharged atomic DM.  There exists a hidden sector where the only coupling to real world
is through photons of term like $FF + F'F' + \epsilon FF'$.
Particles there will appear to have weird charge in our world.
They also fit to CoGeNT anomaly, arXiv 1207.3039.

Sterile neutrino....don't understand at all, but it's worth reading further.

Classically conformal B-L model.  The gauge group here is SM $\times U(1)_{B-L}$,
which will give us right-handed neutrino, a new SM singlet ($\phi$), and the $U(1)_{B-L}$ guy.
New field acquire mass through Yukawa coupling (+ symmetry breaking).
Assign a new $Z_2$ quantum number to right-handed neutrinos (even to first one, odd to others).
The first one is then DM candidate, and the other two are responsible for neutrino oscillation.

\subsection{PDFs, programs, calculations}

A group in DESY is developing relic density calculation at the NLO.
Currently annihilation diagrams are included, but not yet co-annihiliation ones.
See more at their website http://dmnlo.hepforge.org .

NLO calculation is now way better than before.
So...the NLO wish list has been closed recently.
Now, we have NNLO wish list.

In the CTEQ talk difference between several recent CTEQ PDFs are summarized.
CT10 NNLO used pre-LHC data.  Difference between CT10 and CT10w is the inclusion of D0 run-2 Al measurement.
CT12 NLO/NNLO which includes LHC data is coming soon.
One important point he points out is that there is no ``real'' NNLO fit.
Only partial NNLO fit exists so far.


\subsection{Phase transition vs. inflation}

The interesting lesson for me here is the idea of inflaton and waterfall field.
On top of the traditional potential there are a number of different explorations in different directions.
Some explores the (rough) effect of different types of potentials.
This is the first time I heard about this field, so I jotted down some keywords for further reading:
non-gaussianity, tribrid inflation, hybrid inflation, inflaton, waterfall field, Kahler potential,
DBI inflation, tensor to scalar ratio, shift hybrid, smooth hybrid, $\eta$ problem.

\subsection{Variables}

\begin{enumerate}
\item Thrust

\begin{equation}
T = max_{\vec{n}_T} \dfrac{\Sigma |\vec{q}_{T,i} \cdot \vec{n}_T|}{\Sigma |\vec{q}_{T,i}|}  \nonumber
\end{equation}

\item $R_T$, ratio of scalar sum of interesting jets to all jets.

\item Transverse mass calculated using two jets

\item N-jet mass endpoints

\item M3: invariant mass of the three jets that gives maximum total $p_T$, including 1 b-tag and 2 non-tagged
jets.  Useful in top candidates.

\end{enumerate}

\subsection{Standard model}

There's a weird talk of pair of heavy chiral quark annihilating into a lot of longitudinally-polarized
vector bosons.

Precision SM measurement at Jefferson Lab.  From 1994 to 2012, they have completed the 6 GeV run.
They measured electron, proton weak charges (and others).  This might open up window for new physics.
Starting from 2013 (hopefully) they'll move to 11/12 GeV run.

NuTeV collaboration measures the weak angle by first measure charge current to neutral current ratio
of neutrino and anti-neutrino separately, and then infer $sin \theta_W$.
However earlier last decade their preliminary result is weird with 3 $\sigma$ deviation.
The speaker listed few effects that might explain the anomaly: contribution from strange quark,
QED splitting, charge symmetry violation correction, ``EMC effect''.
EMC effect is something no one understands (or so it appears to me), which might be related to
the momentum carried by valence quark compared to momentum carried by nucleon quarks.

\subsection{Higgs-related}

Random points:
\begin{enumerate}
\item Tevatron also see slight excess in photon channel.
\item There's a talk fitting for Higgs coupling.
\item MSSM can accomodate 125 GeV Higgs easily.
\item Higgs in warped extra dimensions might produce modified coupling to W/Z, and maybe photons.
\item One speaker scanned phase spaces for alternate models, and plot out allowed regions for
rate ratio to SM.  The plots are worth checking again.  ``If excess in di-photon exceeds 60\%,
MSSM will be in trouble''
\end{enumerate}

\section{More thought on the milli-charged particle}

First of all, if $\epsilon$ is too small, there's no chance we can see them in colliders.
Sky is the way to go in that case.
In colliders, they will be pair-produced.

Since charge appears to be small, momentum will appear to be large (TeV muons?).
Therefore missing energy will point opposite to the two tracks.
Furthermore dE/dx will appear at weird places with large dE/dx and large momentum.
This could be an interesting quick thing to check.

Depending on DM density around us, the milli-charged particles will have a chance
of interacting with them.  So in colliders we might see such tracks bending for
no good reason.


\section{Summary}

Since I mostly followed the cosmology sessions, I missed most of the SUSY talks.
It's a good overview of the current status of different experiments and what the ``hot''
topics in DM are.
On the Higgs side there's a lot of discussion on the possible excess in the di-photon channel,
but the common conclusion is that we need more data to say for certain what is what.



% \begin{thebibliography}{9}
%    \bibitem{Yay} {\bf Yay Yay Yay},
%       Y. Yayayaya,
%       {\em "Yayayayay study using 2010 data"}
% \end{thebibliography}
 
\pagebreak

\end{document}
