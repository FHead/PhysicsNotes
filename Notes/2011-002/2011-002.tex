\documentclass{fheadnote}

\usepackage{amsmath}
\usepackage{amsfonts}
\usepackage{amssymb}
\usepackage{makeidx}
\usepackage{graphicx}
\usepackage{url}
\usepackage{verbatim}
\usepackage{textcomp}
\usepackage{footnote}
\usepackage{hyperref}
\usepackage{lscape}

\begin{document}

\begin{titlepage}

   \fheadnote{2011/002}
   \date{\today}

   \title{A measurement on b-tagging efficiency in simulation}

   \begin{Authlist}
      Yi Chen
      \Instfoot{Institute}{California Institute of Technology}
   \end{Authlist}

   % \collaboration{FHead}

  \begin{abstract}
  This is a short note documenting how the MC b-tagging efficiency is estimated, gearing towards designing of leptoquark analysis.
  \end{abstract}

\end{titlepage}

\setcounter{page}{2}

\tableofcontents
\clearpage

\section{Background}

In order to use an analysis with b-tagging, we need to know at least qualitatively how the behavior is.
One thing of importance is the b-tagging efficiency as a function of kinematic variables (PT, eta, phi, etc.)
since MR and R rely a lot on the kinematics.


\section{Samples and b-tag thresholds, and technicalities}

The samples we used are the spring11 madgraph TTbar sample.  Statistics/cross section are not important - let's skip it.

B-tags alooked so far and the corresponding thresholds are listed in table \ref{Table_BTagList}.

\begin{table}[htbp]
   \centering
   \begin{tabular}{|c|c|c|c|}
   \hline
   Tag name & Variation & Threshold & Alias \\\hline
   Track counting high efficiency (3D) & Loose & 1.7 & TCHEL \\\hline
   Track counting high efficiency (3D) & Medium & 3.3 & TCHEM \\\hline
   Track counting high efficiency (3D) & Tight & 10.2 & TCHET \\\hline
   Track counting high purity (3D) & Loose & 1.19 & TCHPL \\\hline
   Track counting high purity (3D) & Medium & 1.93 & TCHPM \\\hline
   Track counting high purity (3D) & Tight & 3.41 & TCHPT \\\hline
   Simple secondary vertex high efficiency & Medium & 1.74 & SSVHEL \\\hline
   Simple secondary vertex high efficiency & Tight & 3.05 & SSVHET \\\hline
   Simple secondary vertex high purity & Tight & 2.0 & SSVHPT \\\hline
   \end{tabular}
   \caption{List of tags used in this report}
   \label{Table_BTagList}
\end{table}

\section{Particle-level b hadron tracing and matching}

Since the b quark produced in the hard interaction will undergo different effects (in hadronization step for example) while traversing through the decay chain, final b-jet is sometimes far from the final jet direction.
I have traced the b hadron (or b quark) downstream until the point where it decays weakly, and use this ``final b-hadron'' to do the matching.



\section{Feel of systematics}

\section{Dependence on pileup conditions}

\section{Effects on MR/R shape}

\section{Summary}

% \begin{thebibliography}{9}
%    \bibitem{Yay} {\bf Yay Yay Yay},
%       Y. Yayayaya,
%       {\em "Yayayayay study using 2010 data"}
% \end{thebibliography}
 
\pagebreak

\end{document}

